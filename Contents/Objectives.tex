\section{Objectives}

This literature review shows that, despite substantial advances in both traceability and agentic code-repair systems, the two areas remain largely disconnected. Traditional link-recovery approaches provide robust traceability but often lack explainability and automation; agentic systems (e.g., MAGIS) enable automated fixing but typically operate without explicit traceability grounding. Bridging these paradigms calls for a unified framework that combines interpretable trace links with automated repair capabilities.\\

\noindent
Accordingly, we propose a threefold objective for future research in agentic software maintenance:

\begin{enumerate}
    \item robust \textbf{issue--commit traceability} that explicitly models one-to-many relationships between issues and commits using a learning-to-rank approach; It also involves creating a comprehensive dataset containing one-to-many issue–commit pairs that captures these complex link structures for training and evaluation.
    \item \textbf{explainability mechanisms} that justify link predictions by citing the specific parts of an issue addressed by each commit in natural language; and
    \item \textbf{agentic automation for bug resolution} to propose and, where appropriate, implement fixes guided by traceability evidence and rationales generated by the explainability layer.
\end{enumerate}
\begin{center}
\begingroup
\setlength{\fboxsep}{6pt}% adjust padding as needed
\fbox{%
    \begin{minipage}{\linewidth}
        \textbf{Scope Statement:} This report focuses on and covers the major part of \emph{Task 1: issue--commit traceability} only ,  namely, the design, implementation, and evaluation of explainable, multi-commit issue–commit traceability. Explainability and multi-agent bug fixing frameworks are discussed as future work.
    \end{minipage}%
}
\endgroup
\end{center}

These objectives frame the rest of the report: designing the dataset and feature set, implementing the LinkRank ranking and selection pipeline, evaluating optional semantic features, and validating the approach with a rigorous per-issue protocol to demonstrate its practical benefits for one-to-many issue--commit traceability recovery.\\

\noindent
In the next section, we present \emph{LinkRank}, a learning-to-rank formulation designed specifically for one-to-many issue--commit linking. In LinkRank each issue is treated as a query and candidate commits are ranked using a compact, interpretable feature blend: lexical similarity (TF--IDF with SVD) and retrieval-focused signals (BM25). Ranking is learned with a LambdaMART model and selection uses an iterative pick--remove--renormalize policy. 







