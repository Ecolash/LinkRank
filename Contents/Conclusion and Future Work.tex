\section{Conclusion and Future Work}

Modern software maintenance is hampered by the disconnect between issue reports and the corresponding code changes. This thesis presented the vision for an end-to-end, Agentic AI-driven bug resolution system, conceptualized as a three-stage pipeline: \textbf{Traceability} $\rightarrow$ \textbf{Explainability} $\rightarrow$ \textbf{Resolution}. Such a system promises to automate the entire lifecycle of a bug, from identification to a verified fix. However, the efficacy of the entire pipeline is critically dependent on the quality of its foundation: accurately tracing which commits resolve which issues.\\

\noindent
The central challenge, and the primary focus of this thesis, was that existing traceability research has largely overlooked the common and complex reality of \textit{one-to-many} relationships, where a single issue is resolved by multiple, distinct commits. This gap renders most automated tools insufficient for grounding the high-level reasoning required by an intelligent agent.

\subsection*{Contributions}
In summary, this work contributes the following key items:

\begin{enumerate}
	\item \textit{One-to-many dataset.} We construct a new dataset by mining GitHub pull requests that are linked to \emph{exactly one} issue and contain \emph{two to six} commits, producing genuine one-to-many relations while avoiding degenerate or outlier cases. Careful filtering and repository-aware negative/false-link construction produce a realistic evaluation corpus and reduce ambiguity from multi-issue PRs.

	\item \textit{LinkRank framework.} We cast linking as an issue-centric ranking task and learn a LambdaMART scorer over pairwise features: lexical similarity (TF--IDF{+}SVD) and retrieval focus (BM25). At inference, LinkRank \emph{picks} the top-scoring commit for an issue, \emph{removes} it, \emph{renormalizes} scores among the remaining candidates, and \emph{repeats}. Stopping can be performed with \emph{Known-$K$} (top-$K$) or \emph{Unknown-$K$} (ABS/REL thresholds).

	\item \textit{Optional semantic embeddings.} We add CodeBERT-based semantic similarity as an optional feature channel to test robustness. The marginal improvements observed confirm that LinkRank’s performance is driven primarily by its ranking formulation and IR-style features, which keeps the method efficient and less dependent on transformers.

	\item \textit{LinkRank-C2I variant.} We introduce a bidirectional refinement pipeline: first shortlist issues per commit (commit$\rightarrow$issue ranking), then validate from the issue side (issue$\rightarrow$commit re-ranking) using the same iterative selection policy. This cross-check improves precision while preserving recall and complements the primary formulation.

	\item \textit{Issue-wise (macro) evaluation protocol.} For one-to-many linking we evaluate per-issue by comparing the predicted set $\widehat{\mathcal{C}}(i)$ to the gold set $\mathcal{C}^\star(i)$ using set-based Precision, Recall, and F1, and report macro averages across issues. This directly measures completeness and avoids the optimism of pairwise link-level scoring.
\end{enumerate}

\subsection*{Future Work: Building the Agentic Pipeline}

This thesis has successfully established the first and most critical pillar of the proposed agentic system. By providing a high-fidelity traceability layer, we have laid the necessary groundwork for the subsequent phases of Explainability and Resolution. Future work will build directly upon the accurate one-to-many links recovered by LinkRank.

    \paragraph{Phase 2: Explainability:} The next logical step is to move from \textit{what} (the links) to \textit{why} (the rationale). We envision a new model, likely a fine-tuned Large Language Model or a similar architecture, that takes an issue description and the complete set of its resolving commits (as identified by LinkRank) as input. Its goal would be to analyze the code diffs in the context of the issue and generate a concise, natural-language explanation of the bug's root cause and the logic of the applied multi-commit solution. The main goal of Explainability is to identify and reason which code changes contribute to which aspects of the issue resolution, thereby providing human-understandable justifications for the automated fixes.

    \paragraph{Phase 3: Resolution:} With a comprehensive understanding of issues, linked commits, and their explanations, the final stage is to build an autonomous resolution agent. This agent would leverage the patterns learned in the first two phases to address new, unseen issues. Given a new bug report, the agent would first retrieve similar explained-and-resolved issues, then reason about the context of the new bug, and finally \emph{propose candidate code patches} or even a sequence of commits to resolve it. This could involve generating code diffs directly or suggesting modifications to existing code, effectively closing the loop from issue identification to autonomous resolution.


\noindent
\\The ultimate goal is to integrate all three components---Traceability (LinkRank), Explainability (LLM-based rationale generation), and Resolution (patch generation)---into a single, closed-loop Agentic AI framework. This system would be powerful enough to automatically trigger on new issues, trace the relevant commits, explain the reasoning, and propose fixes with minimal human intervention, revolutionizing the software maintenance landscape. In conclusion, this work has solved a long-standing and critical gap in software traceability. By delivering a robust solution for the one-to-many linking problem, this thesis provides the essential foundation required to build the next generation of truly intelligent and autonomous software maintenance agents.
